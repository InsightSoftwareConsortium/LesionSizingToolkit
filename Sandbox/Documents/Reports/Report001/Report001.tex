\documentclass{InsightArticle}

\usepackage[dvips]{graphicx}

%%%%%%%%%%%%%%%%%%%%%%%%%%%%%%%%%%%%%%%%%%%%%%%%%%%%%%%%%%%%%%%%%%
%
%  hyperref should be the last package to be loaded.
%
%%%%%%%%%%%%%%%%%%%%%%%%%%%%%%%%%%%%%%%%%%%%%%%%%%%%%%%%%%%%%%%%%%
\usepackage[dvips,
bookmarks,
bookmarksopen,
backref,
colorlinks,linkcolor={blue},citecolor={blue},urlcolor={blue},
]{hyperref}

\title{Lung Lesion Segmentation Results}

\release{1.00}

\author{Karthik Krishnan, Luis Ibanez, Wes Turner, Harvey Cline, Rick Avila}
\authoraddress{Kitware Inc., Clifton Park, NY}

\date{\today}

\begin{document}

\ifpdf
\else
   %
   % Commands for including Graphics when using latex
   % 
   \DeclareGraphicsExtensions{.eps,.jpg,.gif,.tiff,.bmp,.png}
   \DeclareGraphicsRule{.jpg}{eps}{.jpg.bb}{`convert #1 eps:-}
   \DeclareGraphicsRule{.gif}{eps}{.gif.bb}{`convert #1 eps:-}
   \DeclareGraphicsRule{.tiff}{eps}{.tiff.bb}{`convert #1 eps:-}
   \DeclareGraphicsRule{.bmp}{eps}{.bmp.bb}{`convert #1 eps:-}
   \DeclareGraphicsRule{.png}{eps}{.png.bb}{`convert #1 eps:-}
\fi

\newcommand{\composeFigureFromDatasetFeatures}[1]{
\begin{figure}
\center
\includegraphics[width=0.7\textwidth]{SCRNFG_#1.png}
\itkcaption[Dataset #1 Features]{Dataset #1 Results of Feature Generators.}
\label{fig:Dataset#1Features}
\end{figure}
\clearpage
}

\newcommand{\composeFigureFromDatasetSegmentations}[1]{
\begin{figure}
\center
\includegraphics[width=0.7\textwidth]{SCRNLSM_#1.png}
\itkcaption[Dataset #1 Segmentations]{Dataset #1 Results of Segmentations.}
\label{fig:Dataset#1Features}
\end{figure}
\clearpage
}


\maketitle

\ifhtml
\chapter*{Front Matter\label{front}}
\fi


\begin{abstract}
\noindent
This document summarizes the results of lung lesion segmentation methods
applied to chest CT scans datasets. In particular, this report focuses on the
behavior of the segmentation method when applied to a region of interest around
a specific lesion.

The figures presented in this report are generated by running the testing suite
over the dataset collection, and therefore should be reproducible in any
computer that has access to that collection.
\end{abstract}

\tableofcontents

\section{Introduction}

This report presents the results of applying the algorithm to a collection of
datasets. The results are presented both as a visual rendering of the resulting
segmentations and by providing quantitative estimations of lesion volume.

\section{Algorithm Description}

\section{Segmentation Results Visualizations}

\composeFigureFromDatasetFeatures{SL0074}

\composeFigureFromDatasetSegmentations{SL0074}

\small
\listoffigures
\listoftables
\normalsize

\end{document}

