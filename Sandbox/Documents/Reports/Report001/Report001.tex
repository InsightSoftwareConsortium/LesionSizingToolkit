\documentclass{InsightArticle}

\usepackage[dvips]{graphicx}

%%%%%%%%%%%%%%%%%%%%%%%%%%%%%%%%%%%%%%%%%%%%%%%%%%%%%%%%%%%%%%%%%%
%
%  hyperref should be the last package to be loaded.
%
%%%%%%%%%%%%%%%%%%%%%%%%%%%%%%%%%%%%%%%%%%%%%%%%%%%%%%%%%%%%%%%%%%
\usepackage[dvips,
bookmarks,
bookmarksopen,
backref,
colorlinks,linkcolor={blue},citecolor={blue},urlcolor={blue},
]{hyperref}

\title{Lung Lesion Segmentation Results}

\release{1.00}

\author{Karthik Krishnan, Luis Ibanez, Wes Turner, Harvey Cline, Rick Avila}
\authoraddress{Kitware Inc., Clifton Park, NY}

\date{\today}

\begin{document}

\ifpdf
\else
   %
   % Commands for including Graphics when using latex
   % 
   \DeclareGraphicsExtensions{.eps,.jpg,.gif,.tiff,.bmp,.png}
   \DeclareGraphicsRule{.jpg}{eps}{.jpg.bb}{`convert #1 eps:-}
   \DeclareGraphicsRule{.gif}{eps}{.gif.bb}{`convert #1 eps:-}
   \DeclareGraphicsRule{.tiff}{eps}{.tiff.bb}{`convert #1 eps:-}
   \DeclareGraphicsRule{.bmp}{eps}{.bmp.bb}{`convert #1 eps:-}
   \DeclareGraphicsRule{.png}{eps}{.png.bb}{`convert #1 eps:-}
\fi

\newcommand{\composeFigureFromDatasetFeatures}[1]{

\subsubsection{Feature Generator Results for Dataset #1}

\begin{figure}
\center
\includegraphics[width=1.0\textwidth]{GMSFGTest#1.png}
\itkcaption[Dataset #1 Gradient Magnitude Sigmoid Feature]{Dataset #1 Results of Gradient Magnitude Sigmoid Feature Generator.}
\label{fig:Dataset#1GMSFG}
\end{figure}
\clearpage

\begin{figure}
\center
\includegraphics[width=1.0\textwidth]{SFGTest#1.png}
\itkcaption[Dataset #1 Sigmoid Feature]{Dataset #1 Results of Sigmoid Feature Generator.}
\label{fig:Dataset#1SFG}
\end{figure}
\clearpage

\begin{figure}
\center
\includegraphics[width=1.0\textwidth]{LWFGTest#1.png}
\itkcaption[Dataset #1 Lung Wall Feature]{Dataset #1 Results of Lung Wall Feature Generator.}
\label{fig:Dataset#1LWFG}
\end{figure}
\clearpage

\begin{figure}
\center
\includegraphics[width=1.0\textwidth]{SVFGTest#1.png}
\itkcaption[Dataset #1 Sato Vesselness Feature]{Dataset #1 Results of Sato Vesselness Feature Generator.}
\label{fig:Dataset#1SVFG}
\end{figure}
\clearpage

\begin{figure}
\center
\includegraphics[width=1.0\textwidth]{SVSFGTest#1.png}
\itkcaption[Dataset #1 Sato Vesselness Sigmoid Feature]{Dataset #1 Results of Sato Vesselness Sigmoid Feature Generator.}
\label{fig:Dataset#1SVSFG}
\end{figure}
\clearpage

\begin{figure}
\center
\includegraphics[width=1.0\textwidth]{SLSFGTest#1.png}
\itkcaption[Dataset #1 Sato Local Structure Feature]{Dataset #1 Results of Sato Local Structure Feature Generator.}
\label{fig:Dataset#1SLSFG}
\end{figure}
\clearpage

\begin{figure}
\center
\includegraphics[width=1.0\textwidth]{DSFGTest#1.png}
\itkcaption[Dataset #1 Descoteaux Sheetness Feature]{Dataset #1 Results of Descoteaux Sheetness Feature Generator.}
\label{fig:Dataset#1DSFG}
\end{figure}
\clearpage

\begin{figure}
\center
\includegraphics[width=1.0\textwidth]{FTFGTest#1.png}
\itkcaption[Dataset #1 Frangi Tubularness Feature]{Dataset #1 Results of Frangi Tubularness Feature Generator.}
\label{fig:Dataset#1FTFG}
\end{figure}
\clearpage

\begin{figure}
\center
\includegraphics[width=1.0\textwidth]{SCRNFG_#1.png}
\itkcaption[Dataset #1 Features]{Dataset #1 Results of Feature Generators.}
\label{fig:Dataset#1Features}
\end{figure}
\clearpage
}

\newcommand{\composeFigureFromDatasetSegmentations}[1]{

\subsubsection{Segmentation Results for Dataset #1}

\begin{figure}
\center
\includegraphics[width=1.0\textwidth]{LSMT3_Test#1.png}
\itkcaption[Dataset #1 Segmentation Method 3]{Dataset #1 Results of Segmentation Method 3.}
\label{fig:Dataset#1LSMT3}
\end{figure}
\clearpage

\begin{figure}
\center
\includegraphics[width=1.0\textwidth]{LSMT4_Test#1.png}
\itkcaption[Dataset #1 Segmentation Method 4]{Dataset #1 Results of Segmentation Method 4.}
\label{fig:Dataset#1LSMT4}
\end{figure}
\clearpage

\begin{figure}
\center
\includegraphics[width=1.0\textwidth]{LSMT5_Test#1.png}
\itkcaption[Dataset #1 Segmentation Method 5]{Dataset #1 Results of Segmentation Method 5.}
\label{fig:Dataset#1LSMT5}
\end{figure}
\clearpage

\begin{figure}
\center
\includegraphics[width=1.0\textwidth]{LSMT6_Test#1.png}
\itkcaption[Dataset #1 Segmentation Method 6]{Dataset #1 Results of Segmentation Method 6.}
\label{fig:Dataset#1LSMT6}
\end{figure}
\clearpage

\begin{figure}
\center
\includegraphics[width=1.0\textwidth]{LSMT7_Test#1.png}
\itkcaption[Dataset #1 Segmentation Method 7]{Dataset #1 Results of Segmentation Method 7.}
\label{fig:Dataset#1LSMT7}
\end{figure}
\clearpage

\begin{figure}
\center
\includegraphics[width=1.0\textwidth]{SCRNLSM_#1.png}
\itkcaption[Dataset #1 Segmentations]{Dataset #1 Results of Segmentations.}
\label{fig:Dataset#1Segmentations}
\end{figure}
\clearpage
}

\newcommand{\insertResultsForDataset}[1]{
\subsection{Dataset #1}
\composeFigureFromDatasetFeatures{#1}
\composeFigureFromDatasetSegmentations{#1}
}

\maketitle

\ifhtml
\chapter*{Front Matter\label{front}}
\fi


\begin{abstract}
\noindent
This document summarizes the results of lung lesion segmentation methods
applied to chest CT scans datasets. In particular, this report focuses on the
behavior of the segmentation method when applied to a region of interest around
a specific lesion.

The figures presented in this report are generated by running the testing suite
over the dataset collection, and therefore should be reproducible in any
computer that has access to that collection.
\end{abstract}

\tableofcontents

\section{Introduction}

This report presents the results of applying the algorithm to a collection of
datasets. The results are presented both as a visual rendering of the resulting
segmentations and by providing quantitative estimations of lesion volume.

\section{Algorithm Description}

\section{Segmentation Results Visualizations}

\insertResultsForDataset{SL0074}
\insertResultsForDataset{SM0052}

\small
\listoffigures
\listoftables
\normalsize

\end{document}

